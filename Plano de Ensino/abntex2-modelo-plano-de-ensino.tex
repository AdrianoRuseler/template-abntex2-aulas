%% abtex2-modelo-artigo.tex, v-1.9.5 laurocesar
%% Copyright 2012-2015 by abnTeX2 group at http://www.abntex.net.br/ 
%%
%% This work may be distributed and/or modified under the
%% conditions of the LaTeX Project Public License, either version 1.3
%% of this license or (at your option) any later version.
%% The latest version of this license is in
%%   http://www.latex-project.org/lppl.txt
%% and version 1.3 or later is part of all distributions of LaTeX
%% version 2005/12/01 or later.
%%
%% This work has the LPPL maintenance status `maintained'.
%% 
%% The Current Maintainer of this work is the abnTeX2 team, led
%% by Lauro César Araujo. Further information are available on 
%% http://www.abntex.net.br/
%%
%% This work consists of the files abntex2-modelo-artigo.tex and
%% abntex2-modelo-references.bib
%%

% ------------------------------------------------------------------------
% ------------------------------------------------------------------------
% abnTeX2: Modelo de Artigo Acadêmico em conformidade com
% ABNT NBR 6022:2003: Informação e documentação - Artigo em publicação 
% periódica científica impressa - Apresentação
% ------------------------------------------------------------------------
% ------------------------------------------------------------------------

\documentclass[
	% -- opções da classe memoir --
	article,			% indica que é um artigo acadêmico
	12pt,				% tamanho da fonte
	twoside,			% para impressão apenas no verso. Oposto a twoside
	a4paper,			% tamanho do papel. 
	% -- opções da classe abntex2 --
	%chapter=TITLE,		% títulos de capítulos convertidos em letras maiúsculas
	%section=TITLE,		% títulos de seções convertidos em letras maiúsculas
	%subsection=TITLE,	% títulos de subseções convertidos em letras maiúsculas
	%subsubsection=TITLE % títulos de subsubseções convertidos em letras maiúsculas
	% -- opções do pacote babel --
	english,			% idioma adicional para hifenização
	brazil,				% o último idioma é o principal do documento
	sumario=tradicional
]{abntex2-modelo-plano-de-aula}


% ---
% PACOTES
% ---

% ---
% Pacotes fundamentais 
% ---
\usepackage{lmodern}			% Usa a fonte Latin Modern
\usepackage[T1]{fontenc}		% Selecao de codigos de fonte.
\usepackage[utf8]{inputenc}		% Codificacao do documento (conversão automática dos acentos)
\usepackage{indentfirst}		% Indenta o primeiro parágrafo de cada seção.
\usepackage{nomencl} 			% Lista de simbolos
\usepackage{color}				% Controle das cores
\usepackage{graphicx}			% Inclusão de gráficos
\usepackage{microtype} 			% para melhorias de justificação
\usepackage{datetime}           % Tempo e hora.
\usepackage{epstopdf}
% ---
		
% ---
% Pacotes adicionais, usados apenas no âmbito do Modelo Canônico do abnteX2
% ---
\usepackage{lipsum}				% para geração de dummy text
% ---
		
% ---
% Pacotes de citações
% ---
\usepackage[brazilian,hyperpageref]{backref}	 % Paginas com as citações na bibl
\usepackage[alf]{abntex2cite}	% Citações padrão ABNT
% ---

\usepackage{listings}
\usepackage{amsmath}		
\usepackage{amssymb}
\usepackage{mathrsfs}
\usepackage{booktabs} % Para Tabelas
\usepackage{subfig}  % permite ter subfiguras
\usepackage{float}
\usepackage{tikz,pgfplots}
\usepackage{pdfpages}
\usepackage{longtable}
\usepackage{framed}
\usepackage{lscape}


\usepackage[framemethod=tikz]{mdframed}
\usepackage{showexpl}
\mdfdefinestyle{mdfexample2}{roundcorner=6pt,backgroundcolor = gray!35}

% ---
\newcommand{\sgn}{\mathop{\mathrm{sgn}}}
\DeclareMathOperator{\deriv}{d}		



\newcounter{NumberInTable}
\newcommand{\LTNUM}{\stepcounter{NumberInTable}{(\theNumberInTable)}}

\newcommand{\Laplace}[1]{\ensuremath{\mathcal{L}{\left[#1\right]}}}
\newcommand{\InvLap}[1]{\ensuremath{\mathcal{L}^{-1}{\left[#1\right]}}}



% Definição de cores
\definecolor{mygreen}{rgb}{0,0.6,0}
\definecolor{mygray}{rgb}{0.5,0.5,0.5}
\definecolor{mymauve}{rgb}{0.58,0,0.82}

\definecolor{shadecolor}{rgb}{0.8,0.8,0.8}

\lstset{ %
	aboveskip=3mm,
	belowskip=3mm,
	backgroundcolor=\color{white},   % choose the background color; you must add \usepackage{color} or \usepackage{xcolor}
	basicstyle={\small\ttfamily},        % the size of the fonts that are used for the code
	breakatwhitespace=true,         % sets if automatic breaks should only happen at whitespace
	breaklines=true,                 % sets automatic line breaking
	captionpos=t,                    % sets the caption-position to bottom
	commentstyle=\color{mygreen},    % comment style
	columns=flexible,
	deletekeywords={...},            % if you want to delete keywords from the given language
	escapeinside={\%*}{*)},          % if you want to add LaTeX within your code
	extendedchars=true,              % lets you use non-ASCII characters; for 8-bits encodings only, does not work with UTF-8
	frame=tb,                        % adds a frame around the code
	keepspaces=true,                 % keeps spaces in text, useful for keeping indentation of code (possibly needs columns=flexible)
	keywordstyle=\color{blue},       % keyword style
	language=Matlab,                 % the language of the code
	morekeywords={*,...},            % if you want to add more keywords to the set
	numbers=none,                    % where to put the line-numbers; possible values are (none, left, right)
	numbersep=5pt,                   % how far the line-numbers are from the code
	numberstyle=\tiny\color{mygray}, % the style that is used for the line-numbers
	rulecolor=\color{black},         % if not set, the frame-color may be changed on line-breaks within not-black text (e.g. comments (green here))
	showspaces=false,                % show spaces everywhere adding particular underscores; it overrides 'showstringspaces'
	showstringspaces=false,          % underline spaces within strings only
	showtabs=false,                  % show tabs within strings adding particular underscores
	stepnumber=2,                    % the step between two line-numbers. If it's 1, each line will be numbered
	stringstyle=\color{mymauve},     % string literal style
	tabsize=3,                       % sets default tabsize to 3 spaces
	texcl=true,						 % Permite o uso de acentuação no código
	title=\lstname                   % show the filename of files included with \lstinputlisting; also try caption instead of title
}

%By default, listings does not support multi-byte encoding for source code. The extendedchar option only works for 8-bits encodings such as latin1.
%
%To handle UTF-8, you should tell listings how to interpret the special characters by defining them like so
\lstset{literate=	
	{á}{{\'a}}1 {é}{{\'e}}1 {í}{{\'i}}1 {ó}{{\'o}}1 {ú}{{\'u}}1
	{Á}{{\'A}}1 {É}{{\'E}}1 {Í}{{\'I}}1 {Ó}{{\'O}}1 {Ú}{{\'U}}1
	{à}{{\`a}}1 {è}{{\`e}}1 {ì}{{\`i}}1 {ò}{{\`o}}1 {ù}{{\`u}}1
	{À}{{\`A}}1 {È}{{\'E}}1 {Ì}{{\`I}}1 {Ò}{{\`O}}1 {Ù}{{\`U}}1
	{ä}{{\"a}}1 {ë}{{\"e}}1 {ï}{{\"i}}1 {ö}{{\"o}}1 {ü}{{\"u}}1
	{Ä}{{\"A}}1 {Ë}{{\"E}}1 {Ï}{{\"I}}1 {Ö}{{\"O}}1 {Ü}{{\"U}}1
	{â}{{\^a}}1 {ê}{{\^e}}1 {î}{{\^i}}1 {ô}{{\^o}}1 {û}{{\^u}}1
	{Â}{{\^A}}1 {Ê}{{\^E}}1 {Î}{{\^I}}1 {Ô}{{\^O}}1 {Û}{{\^U}}1	 
	{œ}{{\oe}}1 {Œ}{{\OE}}1 {æ}{{\ae}}1 {Æ}{{\AE}}1 {ß}{{\ss}}1
	{ű}{{\H{u}}}1 {Ű}{{\H{U}}}1 {ő}{{\H{o}}}1 {Ő}{{\H{O}}}1
	{ç}{{\c c}}1 {Ç}{{\c C}}1 {ø}{{\o}}1 {å}{{\r a}}1 {Å}{{\r A}}1
	{€}{{\EUR}}1 {£}{{\pounds}}1 {ã}{{\~a}}1 {õ}{{\~o}}1 {Ã}{{\~A}}1 {Õ}{{\~O}}1	
}

\renewcommand{\lstlistingname}{Código--fonte }% Listing -> Algorithm
\renewcommand{\lstlistlistingname}{Lista de códigos--fonte}% List of Listings -> List of Algorithms

% ---
% Configurações do pacote backref
% Usado sem a opção hyperpageref de backref
\renewcommand{\backrefpagesname}{Citado na(s) página(s):~}
% Texto padrão antes do número das páginas
\renewcommand{\backref}{}
% Define os textos da citação
\renewcommand*{\backrefalt}[4]{
	\ifcase #1 %
	Nenhuma citação no texto.%
	\or
	Citado na página #2.%
	\else
	Citado #1 vezes nas páginas #2.%
	\fi}%
% ---


% ---
% Informações de dados para CAPA e FOLHA DE ROSTO
% ---
\universidade{Universidade Federal de Santa Catarina}
\centro{CTC -- Centro Tecnológico}
\departamento{Departamento de Engenharia Elétrica e Eletrônica}
\local{Florianópolis}
\data{\today}

\autor{Prof. Nome \textsc{Sobrenome}, Dr. Eng.}

\tipotrabalho{Plano de Ensino}
\disciplina{Eletrônica Industrial}
\codigo{EEL 7278} % Código da disciplina
\semestre{2015/2}
\aula{Plano de Ensino}
\titulo{Modelo Canônico de Notas de Aula com \abnTeX}

\preambulo{Modelo canônico de plano de ensino em conformidade
	com as normas ABNT apresentado à comunidade de usuários \LaTeX.}

\email{adriano.ruseler@posgrad.ufsc.br}
\telefone{Tel: +55 (48) 3721-7464, 3721-7445}
\celular{Cel: +55 (48) 88459814, 91273585}
\website{\url{http://www.adrianoruseler.com}}
\laboratorio{INEP -- Instituto de Eletrônica de Potência}
\campus{Campus Universitário – Trindade, Florianópolis}
\turma{Minha turma}
\horario{Meu horário}
\sala{Minha sala}

% ---
% Configurações de aparência do PDF final

% alterando o aspecto da cor azul
\definecolor{blue}{RGB}{41,5,195}

% informações do PDF
\makeatletter
\hypersetup{
     	%pagebackref=true,
		pdftitle={\@title}, 
		pdfauthor={\@author},
    	pdfsubject={Modelo de plano de ensino com abnTeX2},
	    pdfcreator={LaTeX with abnTeX2},
		pdfkeywords={abnt}{latex}{abntex}{abntex2}{notas de aula}, 
		colorlinks=true,       		% false: boxed links; true: colored links
    	linkcolor=black,          	% color of internal links
    	citecolor=black,        		% color of links to bibliography
    	filecolor=black,      		% color of file links
		urlcolor=black,
		bookmarksdepth=4
}
\makeatother
% --- 

% ---
% compila o indice
% ---
\makeindex
% ---

% ---
% Altera as margens padrões
% ---
\setlrmarginsandblock{3cm}{3cm}{*}
\setulmarginsandblock{3cm}{3cm}{*}
\checkandfixthelayout
% ---


% --- 
% Espaçamentos entre linhas e parágrafos 
% --- 

% O tamanho do parágrafo é dado por:
\setlength{\parindent}{1.3cm}

% Controle do espaçamento entre um parágrafo e outro:
\setlength{\parskip}{0.2cm}  % tente também \onelineskip

% Espaçamento simples
\SingleSpacing

% ----
% Início do documento
% ----
\begin{document}

% Seleciona o idioma do documento (conforme pacotes do babel)
%\selectlanguage{english}
\selectlanguage{brazil}

% Retira espaço extra obsoleto entre as frases.
\frenchspacing 


\imprimircapaUFSC 

%\imprimirletterUFSC

% ]  				% FIM DE ARTIGO EM DUAS COLUNAS
% ---




% ----------------------------------------------------------
% ELEMENTOS TEXTUAIS
% ----------------------------------------------------------
\textual
\pagestyle{notasUFSC}



\begin{mdframed}[style=mdfexample2]	\center
	\section{Dados da Disciplina} % a serem alcançados pelos alunos e não pelo professor. Podem ser divididos em gerais e específicos. 
\end{mdframed}

\begin{description}
	\item[Nome:]  Eletrônica Industrial
	\item[Código:] EEL7278\footnote{ Disciplina fictícia elaborada para o concurso simplificado (Edital n. 047/DDP/2015).} 
	\item[Professor:] Adriano Ruseler
	\item[Curso:] Engenharia Elétrica
	\item[Tipo:] Disciplina fictícia elaborada para o concurso simplificado  \\(Edital n. 047/DDP/2015).
	\item[Carga horária:] 4 créditos
	\item[Pré-requisitos:] EEL7055	-- Circuitos Elétricos B
	\item[Comunicação:] Moodle \url{http://moodle.adrianoruseler.com/}
	\item[Monitoria: ] Disciplina não possui monitor. Em caso de dúvidas, procurar o professor. 
	\item[Ementa:]Retificadores monofásicos e trifásicos a diodo; retificadores monofásicos e trifásicos a tiristor; conversor CC-CC abaixador de tensão; conversor CC-CC elevador; conversores CC-CA de tensão; conversores CC-CA de corrente;
\end{description}



\begin{mdframed}[style=mdfexample2]	\center
	\section{Objetivos} % a serem alcançados pelos alunos e não pelo professor. Podem ser divididos em gerais e específicos. 
\end{mdframed}

Os principais objetivos da disciplina são:

\begin{enumerate}
	\item Introduzir os conceitos fundamentais dos retificadores a diodo e a tiristor;	
	\item Apresentar o princípio de funcionamento das principais topologias retificadoras, e uma metodologia de cálculo para projeto das mesmas.
	\item Introduzir os conceitos fundamentais dos conversores CC-CC e CC-CA;
	\item Apresentar o princípio de funcionamento das principais topologias dos
	conversores CC-CC e CC-CA, e uma metodologia de cálculo para projeto das
	mesmas.
\end{enumerate}

No final do curso o estudante deverá ser capaz de realizar o projeto completo
de um conversor estático.

\hspace{1cm}

\begin{mdframed}[style=mdfexample2]	\center
	\section{Conteúdo programático} % conteúdos programados para a aula organizados em tópicos (de 4 a 8).
\end{mdframed}

\begin{enumerate}      	
	\item\textbf{Retificadores a Diodo}
	\begin{enumerate}
		\item Monofásico de onda completa
		\item Trifásico de onda completa
		\item Trifásico com ponto médio
	\end{enumerate}
	\item\textbf{Retificadores a Tiristor}
	\begin{enumerate}
		\item Monofásico de onda completa
		\item Trifásico de onda completa
		\item Trifásico com ponto médio
	\end{enumerate}       		
	
	
	\item \textbf{Conversores CC-CC abaixador de tensão (Buck)}
	\begin{enumerate}
		\item Princípio de operação	
		\item Funcionamento com carga RLE
		\item Condução contínua e descontínua
		\item Característica de carga
		\item Ondulação da corrente
		\item Filtragem (corrente de entrada e tensão de saída)
		\item Controle do conversor Buck empregando modulação PWM
		\item Conversor Buck isolado (Conversor Forward)  	
	\end{enumerate}  
	
	\item \textbf{Conversor CC-CC elevador (Boost)}
	\begin{enumerate}
		\item Princípio de operação
		\item Condução Contínua e descontínua
		\item Característica de carga
		\item Ondulação da corrente
		\item Filtros de entrada e de saída
		\item Controle do conversor Boost empregando modulação PWM
	\end{enumerate}        		
	
	
	\item\textbf{Conversores CC-CA de tensão}       
	
	\begin{enumerate}
		\item Conversor CC-CA monofásico em ponte
		\item  Conversor CC-CA monofásico com ponto médio
		\item  Conversor CC-CA trifásico
		\item  Reversibilidade dos conversores CC-CA de tensão
	\end{enumerate} 
	
	\item \textbf{Conversor CC-CA de corrente}
	\begin{enumerate}
		\item  Conversor CC-CA de corrente monofásico
		\item  Conversor CC-CA de corrente trifásico
	\end{enumerate}
	
	
	
\end{enumerate} 




\begin{mdframed}[style=mdfexample2]	\center
	\section{Ferramentas utilizadas na disciplina}
\end{mdframed}        

Serão utilizados dois aplicativos no decorrer da disciplina:
\begin{enumerate}
	\item \textbf{PSIM} - Simulação numérica 
	\url{http://powersimtech.com/download-demo/}
	\item \textbf{MATHCAD} - Planilha de cálculos \url{http://www.ptc.com/product/mathcad}
\end{enumerate} 

Treinamento básico dos softwares utilizados se dará no decorrer das aulas.
Maiores informações podem ser encontradas no site da disciplina \url{http://eel7278.adrianoruseler.com} 

\begin{mdframed}[style=mdfexample2]	\center
	\section{Avaliação} % pode ser realizada com diferentes propósitos (diagnóstica, formativa e somativa). Interessante explicitar a atividade avaliativa e os critérios de correção.
\end{mdframed}

O aluno será avaliado através de 2 provas ($P1$ e $P2$) e um trabalho de projeto e simulação proposto ($T1$).
A média final $MF$ será calculada da seguinte maneira:
\[
MF = \left( {\frac{{P1 + P2 + T1}}{3}} \right)
\]

Critério de aprovação:
\[
\begin{cases}
\textbf{Aprovado},& \text{se   } MF \ge 6,00;\\
\textbf{Recuperação},& \text{se   } 3,00 \le MF < 6,00;\\
\textbf{Reprovado}, & \text{se   } MF \le 3,00
\end{cases}
\]

Caso o aluno necessite da prova de recuperação (REC), a média final com recuperação (MFR) será: 

\[MFR = \frac{{MF + REC}}{2}\]

Critério final de aprovação:
\[
\begin{cases}
\textbf{Aprovado},& \text{se   } MFR \ge 6,00;\\
\textbf{Reprovado}, & \text{se   } MFR \le 6,00
\end{cases}
\]


\newpage
\textbf{Conteúdo da Prova 01:}
\begin{enumerate}      	
	\item {Retificadores a Diodo}
	\begin{enumerate}
		\item Monofásico de onda completa
		\item Trifásico de onda completa
		\item Trifásico com ponto médio
	\end{enumerate}
	\item Retificadores a Tiristor
	\begin{enumerate}
		\item Monofásico de onda completa
		\item Trifásico de onda completa
		\item Trifásico com ponto médio
	\end{enumerate}       		
\end{enumerate}

\textbf{Conteúdo da Prova 02:}  

\begin{enumerate}      	
	
	\item  Conversores CC-CC abaixador de tensão (Buck)    	
	
	\item Conversor CC-CC elevador (Boost)       	
	
	\item Conversores CC-CA de tensão       
	
	\begin{enumerate}
		\item Conversor CC-CA monofásico em ponte
		\item  Conversor CC-CA monofásico com ponto médio
		\item  Conversor CC-CA trifásico
	\end{enumerate} 
	
	\item Conversor CC-CA de corrente
	\begin{enumerate}
		\item  Conversor CC-CA de corrente monofásico
		\item  Conversor CC-CA de corrente trifásico
	\end{enumerate}   	
\end{enumerate} 	




\emph{As datas das provas e dos trabalhos poderão sofrer alterações.} 



\begin{mdframed}[style=mdfexample2]	\center
	\section{Bibliografia}
\end{mdframed}

\subsection{Bibliografia Básica}




I. Barbi, “Eletrônica de Potência”. Edição do Autor, 6a Edição Florianópolis,
2006.

D. C. Martins \& I. Barbi, “Eletrônica de Potência: Conversores CC-CC Básicos
Não Isolados”. Edição dos Autores, 3a Edição Florianópolis, 2008.

D. C. Martins \& I. Barbi, “Introdução ao Estudo dos Conversores CC-CA”.
Edição dos Autores, 2a Edição, Florianópolis-SC, UFSC, Maio/2008.

I. Barbi, “Eletrônica de Potência: Projetos de Fontes Chaveadas”. Edição do
Autor, 2a Edição, Florianópolis-SC, UFSC, 2007.

I. Barbi \& F. P. de Souza, “Conversores CC-CC Isolados de Alta Freqüênica
com Comutação Suave”. Edição dos Autores, Florianópolis, 1999.

D. C. Martins, “Eletrônica de Potência – Semicondutores de Potência
Controlados, Conversores CC-CC Isolados e Conversores CC-CC a Tiristor
(Comutação Forçada)”. Publicação Interna – UFSC-INEP, Florianópolis, SC,
Maio/2006.

A.J. Perin, “Teoria e Aplicação de Modulação por Largura de Pulsos (PWM)
com Otimização de Harmônicas para Conversores Estáticos de Frequência”. 6ª
CBA – Minicursos, pp. 01-15, Belo Horizonte-MG, Novembro/1986.



\subsection{Bibliografia Complementar}
N. Mohan, T. Underland \& W. Robbins, “Power Electronics: Converters,
Applications and Design”. John Wiley \& Sons, New York-USA, 2ª Edição, 1995.

B. W. Williams, “Power Electronics – Devices, Drives, Applications and Passive
Components McGraw-Hill, Inc., New York-USA, 2ª Edição, 1992.

A. I. Pressman, “Switching Power Supply Design”. McGraw-Hill, Inc., New York-
USA, 1991.

R.G. Hoft, “Semiconductor Power Electronics”. Van Nostrand Reinhold
Company. Inc., New York-USA, 1986.

C. W. Lander, “Eletrônica Industrial Teoria e Aplicações”. McGraw-Hill, Rio de
Janeiro, 1988.

M. H. Rashid, “Power Electronics – Circuits, Devices, and Applications”.
Prentice-Hall International Editions, Inc., New Jersey, 1988.

R. W. Erickson, “Fundamentals of Power Electronics”. Editora Chapman \& Hall,
New York, USA, 1997.

J. G. Kassakian, M. F. Schlecht \& G. C. Verghese, “Principles of Power
Electronics“. Addison-Wesley Publishing Company, Inc., Massachussets, USA.

S. B. Dewan, G. R. Slemon \& A. Straughen, “Power Semiconductor Drives”. A
Wiley-Interscience Publication, John Wiley \& Sons, Inc. New York, 1984.

J. Vithayathil, “Power Electronics – Principles and Applications”. McGraw-Hill,
Inc., 1995.

R. S. Ramshaw, “Power Electronics Semiconductors Switches”. Chapman \&
Hall, 2nd Edition, 1993.

B. D. Bedford \& R. G. Hoft, “Principles of Inverter Circuits“. John Wiley \& Sons,
Inc., New York, 1964.

\hspace{1cm}
\newpage
\begin{mdframed}[style=mdfexample2]	\center
	\section{Sítios importantes}
\end{mdframed}

\begin{description}
	\item[Moodle da Disciplina] \url{http://moodle.adrianoruseler.com/}
	\item[Site da disciplina EEL7278] \url{http://eel7278.adrianoruseler.com/}
	\item[Universidade Federal de Santa Catarina] \url{http://ufsc.br/}
	\item[Departamento de Engenharia Elétrica e Eletrônica] \url{http://deel.ufsc.br/}
	\item[Instituto de Eletrônica de Potência] \url{http://inep.sites.ufsc.br/}
	%	\item[Power Guru]\url{http://www.powerguru.org/}
\end{description}




\begin{mdframed}[style=mdfexample2]	\center
	\section{Cronograma de aulas}
\end{mdframed}

\begin{description}
	\item[Aula 01] – Retificador Monofásico de Onda Completa a Diodo;
	\item[Aula 02] – Retificador Trifásico com Ponto Médio a Diodo;
	\item[Aula 03] – Retificador Trifásico de Onda Completa a Diodo;
	\item[Aula 04] – Retificador Monofásico de Onda Completa a Tiristor;
	\item[Aula 05] – Retificador Trifásico com Ponto Médio a Tiristor;
	\item[Aula 06] – Retificador Trifásico de Onda Completa a Tiristor;
	\item[Aula 07] – Conversor CC-CC Abaixador de Tensão (Buck);
	\item[Aula 08] – Conversor CC-CC Elevador de Tensão (Boost);
	\item[Aula 09] – Conversor CC-CA de Tensão;
	\item[Aula 10] – Conversor CC-CA de Corrente.
	
\end{description}



\begin{landscape}
	\thispagestyle{empty}
	% 01202C, 01235B, 01235C, 02213A e 02213B.
	\begin{table}[!h]
		\caption{Cronograma de aulas para a disciplina EEL 7011- Eletricidade Básica, semestre 2015.2.}
		\label{tab:CronogramaEEL7011}
		\centering
		\resizebox{\linewidth}{!}{%
			\begin{tabular}{c|c|c|c|c|c|c} \toprule 
				\multicolumn{2}{c|}{} & Segunda	& \multicolumn{2}{c|}{Terça} & Quinta & Sexta \\ \hline
				Semana       & De -- Até       & 2213B (2.1010-2)        & 2213A (3.0820-2)  & 1235C (3.1830-2)    & 1235B (5.1710-2)         & 1202C (6.1010-2)  \\\hline
				1       & 10/08 -- 14/08       &  \textit{Sem Aula}       & \textit{Sem Aula}  & \textit{Sem Aula}    & \textit{Sem Aula }        & \textit{Sem Aula} \\\hline
				2       & 17/08 -- 21/08       &  Aula 00       & Aula 00  & Aula 00    & Aula 00         & Aula 00 \\\hline
				3       & 24/08 -- 28/08       &  Aula 01       & Aula 01  & Aula 01    & Aula 01         & Aula 01 \\\hline
				4       & 31/08 -- 04/09       &  Aula 02       & Aula 02  & Aula 02    & Aula 02         & Aula 02 \\\hline
				5       & 07/09 -- 11/09     &  \textbf{Feriado}& Aula 03  & Aula 03   & Aula 03         & Aula 03 \\\hline
				6       & 14/09 -- 18/09       &  Aula 03       & Aula 04  & Aula 04    & Aula 04         & Aula 04 \\\hline
				7       & 21/09 -- 25/09       &  Aula 04       & Aula 05  & Aula 05    & Aula 05         & Aula 05 \\\hline
				8       & 28/09 -- 02/10       &  Aula 05       & Aula 06  & Aula 06    & Aula 06         & Aula 06 \\\hline
				9       & 05/10 -- 09/10       &  Aula 06       & Aula 07  & Aula 07    & Aula 07         & Aula 07 \\\hline
				10       & 12/10 -- 16/10     &  \textbf{Feriado}& Aula 08  & Aula 08  & Aula 08        & Aula 08 \\\hline
				11       & 19/10 -- 23/10       &  Aula 07       & Aula 09  & Aula 09    & Aula 09         & Aula 09 \\\hline
				12       & 26/10 -- 30/10       &  Aula 08       & Aula 10  & Aula 10    & Aula 10         & Aula 10 \\\hline
				13       & 02/11 -- 06/11     &  \textbf{Feriado}& Aula 11  & Aula 11  & Aula 11         & Aula 11 \\\hline
				14       & 09/11 -- 13/11       &  Aula 09       & Aula 12  & Aula 12    & Aula 12         & Aula 12 \\\hline
				15       & 16/11 -- 20/11       &  Aula 10       &\textit{ Sem Aula } & \textit{Sem Aula}    & \textit{Sem Aula}        & \textit{Sem Aula} \\\hline
				16       & 23/11 -- 27/11       &  Aula 11       &\textit{ Sem Aula}  & \textit{Sem Aula}    & \textit{Sem Aula}        & \textit{Sem Aula} \\\hline
				17       & 30/11 -- 04/12       &  Aula 12       & \textit{Sem Aula}  & \textit{Sem Aula}    & \textit{Sem Aula}         & \textit{Sem Aula} \\\hline
				18       & 07/12 -- 11/12       &  \textit{Sem Aula}       & \textbf{Feriado}  & \textbf{Feriado}    & \textit{Sem Aula}         & \textit{Sem Aula} \\	
				\bottomrule
			\end{tabular}}
		\end{table}	
		
		\textbf{Turmas:   2213B, 2213A, 1235C, 1235B e 1202C. (Prof. Adriano Ruseler)}.	
		Mais informações: 
		\begin{description}
			\item[Site do Prof.:] \url{http://www.professor.adrianoruseler.com}
			\item[Moodle UFSC:] \url{https://moodle.ufsc.br/}
		\end{description}
		
		
	\end{landscape}	



% Referências bibliográficas
\clearpage
\pagestyle{plain}
\begin{thebibliography}{}
	
	\bibitem{IvoBarbi2006} I. Barbi, “Eletrônica de Potência”. Edição do Autor, 6a Edição Florianópolis,
	2006.
	
	\bibitem{DCMBarbi1} D. C. Martins \& I. Barbi, “Eletrônica de Potência: Conversores CC-CC Básicos Não Isolados”. Edição dos Autores, 3a Edição Florianópolis, 2008.
	
	\bibitem{DCMBarbi2}	D. C. Martins \& I. Barbi, “Introdução ao Estudo dos Conversores CC-CA”. Edição dos Autores, 2a Edição, Florianópolis-SC, UFSC, Maio/2008.
	
	\bibitem{Barbi02} I. Barbi, “Eletrônica de Potência: Projetos de Fontes Chaveadas”. Edição do
	Autor, 2a Edição, Florianópolis-SC, UFSC, 2007.
	
	\bibitem{BarbiFabiana}	I. Barbi \& F. P. de Souza, “Conversores CC-CC Isolados de Alta Freqüênica com Comutação Suave”. Edição dos Autores, Florianópolis, 1999.
	
	\bibitem{DCM}	D. C. Martins, “Eletrônica de Potência – Semicondutores de Potência
	Controlados, Conversores CC-CC Isolados e Conversores CC-CC a Tiristor
	(Comutação Forçada)”. Publicação Interna – UFSC-INEP, Florianópolis, SC,
	Maio/2006.
	
\end{thebibliography}






% ----------------------------------------------------------
% ELEMENTOS PÓS-TEXTUAIS
% ----------------------------------------------------------
% \postextual



\end{document}
