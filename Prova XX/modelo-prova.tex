
\documentclass[11pt,paper=a4,answers]{exam}

\newcommand*{\renameenviron}[1]{%
	\expandafter\let\csname exam-#1\expandafter\endcsname
	\csname #1\endcsname
	\expandafter\let\csname endexam-#1\expandafter\endcsname
	\csname end#1\endcsname
	\expandafter\let\csname #1\endcsname\relax
	\expandafter\let\csname end#1\endcsname\relax
}
\renameenviron{framed}
\renameenviron{shaded}
\renameenviron{leftbar}
\usepackage{framed}

\usepackage[utf8]{inputenc}		% Codificacao do documento (conversão automática dos acentos)
\usepackage{graphicx,lastpage}
\usepackage{upgreek}
\usepackage{censor}
%\usepackage{framed}


\censorruledepth=-.2ex
\censorruleheight=.1ex
\hyphenpenalty 10000
\usepackage[paperheight=10.5in,paperwidth=8.27in,bindingoffset=0in,left=0.8in,right=1in,
top=0.7in,bottom=1in,headsep=.5\baselineskip]{geometry}


\flushbottom
\usepackage[normalem]{ulem}
\renewcommand\ULthickness{2pt}   %%---> For changing thickness of underline
\setlength\ULdepth{1.5ex}%\maxdimen ---> For changing depth of underline
\renewcommand{\baselinestretch}{1}
\pagestyle{empty}

\pagestyle{headandfoot}
\headrule
\newcommand{\continuedmessage}{%
	\ifcontinuation{\footnotesize Question \ContinuedQuestion\ continues\ldots}{}%
}
\runningheader{\footnotesize EEL7278 -- Eletrônica Industrial}
{\footnotesize }
{\footnotesize P\'agina \thepage\ de \numpages}
\footrule
\footer{\footnotesize Prova 01 -- \today}
{}
{\ifincomplete{\footnotesize Pergunta  \IncompleteQuestion\ continua na pr\'oxima p\'agina \ldots}{\iflastpage{\footnotesize Fim do Exame.}{\footnotesize Continua na pr\'oxima p\'agina\ldots}}}

\usepackage{cleveref}
\crefname{figure}{figure}{figures}
\crefname{question}{question}{questions}


%==============================================================
\begin{document}
	
	%% \thispagestyle{empty}	
	\noindent
	\begin{minipage}[l]{.1\textwidth}%
		\noindent
		\includegraphics[width=2\textwidth]{logo-ufsc}
	\end{minipage}
	\hfill
	\begin{minipage}[r]{.8\textwidth}%
		\begin{center}
			{\large \bfseries UNIVERSIDADE FEDERAL DE SANTA CATARINA \par
				\large Departamento de Engenharia Elétrica e Eletrônica \\[4pt]
			%	\vspace{0.5cm}
				Eletrônica Industrial {(EEL7278)}  \par
				\vspace{0.5cm}
				\huge Prova 01
				}
			%  \vspace{0.5cm}
		\end{center}
	\end{minipage}	
		
		\par
		\noindent
		\uline
%		\uline{Tempo: 90 min.   \hfill \normalsize\emph{\underline{Term}} \hfill  }
	\hfill 	
			
\par
\bigskip\bigskip
\begin{minipage}[t]{.6\textwidth}%
	Professor: Adriano Ruseler \par
	Semestre: 2015/2 \par
	Aluno: \makebox[2.5in]{\hrulefill} \par
	
\end{minipage}%
\hfill
\begin{minipage}[t]{.4\textwidth}%
	Turma:  07202A \par
     \par
	Matricula: \makebox[1.5in]{\hrulefill} \par
\end{minipage}
\par
\bigskip


\begin{framed}
	\centering Prova referente aos assuntos:\\ \raggedright
	 \textbf{Aula 01 -- Retificador Monofásico de Onda Completa a Diodo }\\
	 \textbf{Aula 02 – Retificador Monofásico de Onda Completa a Tiristor }\\
	 \textbf{Aula 03 – Retificador Trifásico de Onda Completa a Diodo}\\
	 \textbf{Aula 04 – Retificador Trifásico com Ponto Médio a Diodo}\\
	 \textbf{Aula 05 – Retificador Trifásico de Onda Completa a Tiristor}\\
	 \textbf{Aula 06 – Retificador Trifásico com Ponto Médio a Tiristor}
\end{framed}


		
\begin{questions}
			
\pointsinrightmargin
\pointsdroppedatright
\marksnotpoints
%\marginpointname{mark}
\pointpoints{mark}{marks}
\pointformat{\boldmath\themarginpoints}
\bracketedpoints
			
			
			\question[06]
			\label{Q:perunit}
			For a surface $\vec{r}= \vec{r} (u \cos v, u \sin v, f(u))$. Write down the first fundamental form of the surface. Show that the parametric curves are orthogonal.
			\droppoints
			
			\question[06]
			\label{Q:perunit}
			For a surface $\vec{r}= \vec{r} (u \cos v, u \sin v, f(u))$. Write down the first fundamental form of the surface. Show that the parametric curves are orthogonal.
			\droppoints
			
						
			\question[10]
			\label{Q:zbus}
			Prove that necessary conditions for the curve $u = u(t), v = v(t)$ on a surface $\vec(r) = \vec(r)(u,v)$ to be geodesic is that \begin{equation}U \frac{\partial T}{\partial \dot{v}} - V    \frac{\partial T}{\partial \dot{u}}\end{equation}
			where
			$$ U = \frac{d}{dt} \Big(\frac{\partial T}{\partial \dot{u}}\Big) - \frac{\partial T}{\partial u} = \frac{1}{2T}\frac{dT}{dt}\frac{\partial T}{\partial \dot{u}}$$
			$$ V = \frac{d}{dt} \Big(\frac{\partial T}{\partial    \dot{v}}\Big) - \frac{\partial T}{\partial v} = \frac{1}{2T}\frac{dT}{dt}\frac{\partial T}{\partial \dot{v}}$$
			\droppoints
			
			
			
			
			
			
			
			\question[8]
			\label{Q:zbus}
			For the curve
			$$
			x = a(3u - u^{3}),\qquad y = 3au^{2},\qquad z = a(3u + u^{3})
			$$
			show that $$\uptau = k  =  \frac{1}{3a(1+u^{2})^{2}}$$
			\droppoints
			
			
			
			
			
			
			
			\question[8]
			\label{Q:zbus}
			A curve is uniquely determined except as the position in space, when its curvature and            torsion are given functions of its arc length.
			\droppoints
			
			
			
			\question[8]
			\label{Q:zbus}
			Show that there exists an infinite family of involutes for a  gives curve.
			\droppoints
			\newpage
			
			
			
			
			\question[08]
			\label{Q:ybus}
			Give short answers of the following questions.
			\begin{enumerate}
				\item Define Helicoids?
				\item Define spherical indicatrix?
				\item Define the intrinsic equation?
				\item Write the statement of existence theorem for space curve?
				\item The normal curvature $k_{n}$ is equal to the what?
				\item Prove that $L = -n_{1} \cdot r_{1}$ and $N = -n_{2}    \cdot r_{2}$?
				\item Define the geodesic?
				\item Write down the equation of tangent plane?
				\item If equation of the circle is $x^{2} + y^{2} = a^{2}$ then the parametric equations            of circles are \xblackout{forty     two}?
			\end{enumerate}
		\end{questions}
		\begin{center}
			\rule{.5\textwidth}{1pt}
		\end{center}
	\end{document} 